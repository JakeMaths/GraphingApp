\documentclass{article}
\usepackage[margin=1cm,bottom=2cm,headheight=5\baselineskip,headsep=1\baselineskip,includehead]{geometry}
\usepackage{tabularx}
\usepackage{graphicx}
\usepackage{multirow}

\renewcommand{\title}{Initial Proposal}
\renewcommand{\date}{September 13, 2023}
\newcommand{\authorone}{Jake Mason}
\newcommand{\authortwo}{Upmanyu Rohit}
\newcommand{\classname}{n-Dimensional Graphing Calculator}
\newcommand{\classcode}{CS 4091}
\newcommand{\makevspace}{\vspace{0.5cm}}

%% set up and width for the tabularx environment to expand and fit to.
\newlength{\headerwidth}
\setlength{\headerwidth}{\textwidth}
\newsavebox{\myheader}
\begin{lrbox}{\myheader}%
    \begin{minipage}[b]{\headerwidth}
    	\renewcommand{\arraystretch}{1.29}%
    	\begin{tabularx}{\headerwidth}{|c|X|X|} 
		\hline
               	\multirow{3}{*}{ \hspace{0.25cm} \includegraphics[height=0.5in]{st.png} \hspace{0.25cm}}
			  & \multicolumn{2}{c|}{ \textbf{\title} } \\
		\cline{2-3}
                	  & \centering\textbf{ \classname } & \centering\arraybackslash\textbf{ \classcode } \\
		\cline{2-3}
                	  & \centering\textbf{ \date } & \centering\arraybackslash\textbf{ \authorone \hspace{1pt} \& \authortwo } \\
		\hline
    	\end{tabularx}
    \end{minipage}
\end{lrbox}

%% Setting up the header
\usepackage{fancyhdr}
\pagestyle{fancy}
\renewcommand{\headrulewidth}{0pt}
\renewcommand{\footrulewidth}{0pt}
\lhead{}
\chead{\usebox{\myheader}}
\rhead{}
\lfoot{}
\cfoot{}
\rfoot{\Large \thepage}

\usepackage{lipsum}
\usepackage{amsmath}
\usepackage{amssymb}
\usepackage{amsthm}

\begin{document} {\Large

\section{Problem}
Consider a real valued function of one variable $f:\mathbb{R} \to \mathbb{R}$. By a 
graph $G$ of the function $f$, we mean a subset of $\mathbb{R}^2$,
\[
	G = \{ (x, y) \in \mathbb{R} \times \mathbb{R} : y = f(x) \}.	
\]
There exist many applications for visualizing the graph of single variable functions.
Several hand held calculators can graph single variable functions. 
For example, Texas Instruments has a lineup of graphing calculators including
the leading TI-Inspire. 
There are also online applications such as Desmos and GeoGebra with similar capabilities.

\makevspace

Now consider a real valued function of $n$ variables $f:\mathbb{R}^n \to \mathbb{R}$.
The graph $G$ of $f$ is defined similarly to that of single variable functions,
\[
	G = \{ (\mathbf{x}, y) \in \mathbb{R}^n \times \mathbb{R} : y = f(\mathbf{x}) \}.	
\]
However, this graph is a subset of $\mathbb{R}^{n+1}$.
It is not obvious how one should visualize this graph.
A 3 dimensional world offers efficient visualization of two-variable functions at best. 
There are no well known calculators capable of visualizing a real function of $n$ variables.

\makevspace

The problem is that functions of $n$ variables are common in nature. 
It is hard to find a system where values are only dependent on one variable. 
We seek an application capable of visualzing the graph of functions of $n$ variables.
Such an application will offer details on functions where previously calculations had to 
be carried out by hand.

\newpage
\section{Solution}
We propose the design of a calculator capable of visualizing the graph of a 
real valued function of $n$ variables, where $n\geq 1$. Our goal is to first design an
application with 2 dimensional and 3 dimensional graphing capabilities. We can then 
implement functions of $n$ variables and view them using projections onto 2 or 3 
dimensional subspaces. Consider a function of $n$ variables 
\[
	f: \mathbb{R}^n \to \mathbb{R}.	
\]
We can project this function onto the $x_i, x_j$ plane by setting 
\[
	g: \mathbb{R}^2 \to \mathbb{R} \quad 
	g(x_i, x_j) = f(0, \dots, x_i, \dots, x_j, \dots, 0)	
\]
where $x_i$ and $x_j$ fall in the $i^{th}$ and $j^{th}$ positions respectively.
A similar result holds for projecting onto 3 dimensional space. 
We may also consider projections onto arbitrary planes. 
A plane in 3 dimensional space can be obtained through the equation 
\[
	ax + by + cz = d 	
\]
with $a,b,c,d\in\mathbb{R}$. Projecting a function onto such a plane 
may involve careful rotation of the coordinate axes. 

With the general idea set, we plan to implement this design with a 
user interface for entering functions and selecting a graph view.

\newpage
\section{Milestones \& Tasking}
\begin{tabularx}{\headerwidth}{|c|X|} 
	\hline
	\multirow{3}{*}{ 2D Graphing } & Program Initialization \\
								   & Draw to the screen \\
								   & Input functions of 1 variable \\
	\multirow{3}{*}{ By Sept. 30 } & Store functions of 1 variable \\
								   & Display graph of function \\
								   & Allow translation of graph view \\
	\hline
	\multirow{2}{*}{ 3D Graphing } & Input functions of 2 variables \\
								   & Store functions of 2 variables \\
	\multirow{2}{*}{ By Oct. 31 }  & Display graph (already using projections) \\
								   & Allow rotation of graph view \\ 
	\hline
	\multirow{3}{*}{ nD Graphing } & Input functions of $n$ variables \\
								   & Store functions of $n$ variables \\
								   & Allow choice of projective axes \\
	\multirow{2}{*}{ By Nov. 30 }  & Allow choice of projective plane \\
								   & Display graph onto projected surface \\
	\hline
	\multirow{2}{*}{ Extra } & Calculate properties of functions such as max, min, avg \\
	                         & Display features of graph view such as range, strecth, axes \\
	By Nov. 30				 & Allow saving of graphs for later use \\
	\hline
	Testing & Verify graphs of well known functions \\
	\hline
	\multirow{3}{*}{ Deployment } & Show application to mathematics majors and professors \\
								  & Take feedback and make necessary adjustments \\
								  & Continue support for new product \\
	\hline
\end{tabularx}

\newpage
\section{MOSCOW}
\begin{tabularx}{\headerwidth}{|c|X|} 
	\hline
		  \multirow{5}{*}{ \textbf{M}UST: } 
		& { Method for displaying graphs of single variable functions on screen } \\
		& { Method for storing functions of $n$ variables } \\
		& { Method for displaying projections of $n$ variable functions } \\
		& { Method for user to input $n$ variable functions } \\
		& { Run at stable 60fps with no warnings/errors } \\
	\hline
		\multirow{2}{*}{ \textbf{S}HOULD: }
		& { Parser to allow function input as text } \\
		& { Display properties of function such as max, min, avg } \\
	\hline
		\multirow{2}{*}{ \textbf{C}OULD: }
		& { Run on both Windows and Linux } \\
		& { Allow saving of graphs for later use } \\
	\hline
		\multirow{1}{*}{ \textbf{W}OULD: }
		& { Parser to allow function input as \LaTeX } \\
	\hline
\end{tabularx}

} \end{document}