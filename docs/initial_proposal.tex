\documentclass{article}
\usepackage[margin=1cm,bottom=2cm,headheight=5\baselineskip,headsep=1\baselineskip,includehead]{geometry}
\usepackage{tabularx}
\usepackage{graphicx}
\usepackage{multirow}

\renewcommand{\title}{Initial Proposal}
\renewcommand{\date}{September 13, 2023}
\newcommand{\authorone}{Jake Mason}
\newcommand{\authortwo}{Upmanyu Rohit}
\newcommand{\classname}{n-Dimensional Graphing Calculator}
\newcommand{\classcode}{CS 4091}
\newcommand{\makevspace}{\vspace{0.5cm}}

%% set up and width for the tabularx environment to expand and fit to.
\newlength{\headerwidth}
\setlength{\headerwidth}{\textwidth}
\newsavebox{\myheader}
\begin{lrbox}{\myheader}%
    \begin{minipage}[b]{\headerwidth}
    	\renewcommand{\arraystretch}{1.29}%
    	\begin{tabularx}{\headerwidth}{|c|X|X|} 
		\hline
               	\multirow{3}{*}{ \hspace{0.25cm} \includegraphics[height=0.5in]{st.png} \hspace{0.25cm}}
			  & \multicolumn{2}{c|}{ \textbf{\title} } \\
		\cline{2-3}
                	  & \centering\textbf{ \classname } & \centering\arraybackslash\textbf{ \classcode } \\
		\cline{2-3}
                	  & \centering\textbf{ \date } & \centering\arraybackslash\textbf{ \authorone \hspace{1pt} \& \authortwo } \\
		\hline
    	\end{tabularx}
    \end{minipage}
\end{lrbox}

%% Setting up the header
\usepackage{fancyhdr}
\pagestyle{fancy}
\renewcommand{\headrulewidth}{0pt}
\renewcommand{\footrulewidth}{0pt}
\lhead{}
\chead{\usebox{\myheader}}
\rhead{}
\lfoot{}
\cfoot{}
\rfoot{\Large \thepage}

\usepackage{lipsum}
\usepackage{amsmath}
\usepackage{amssymb}
\usepackage{amsthm}

\begin{document} {\Large

\section{Problem}
Consider a real valued function of one variable $f:\mathbb{R} \to \mathbb{R}$. By a 
graph $G$ of the function $f$, we mean a subset of $\mathbb{R}^2$,
\[
	G = \{ (x, y) \in \mathbb{R} \times \mathbb{R} : y = f(x) \}.	
\]
There exist many applications for visualizing the graph of single variable functions.
Several hand held calculators can graph single variable functions. 
For example, Texas Instruments has a lineup of graphing calculators including
the leading TI-Inspire. 
There are also online applications such as Desmos and GeoGebra with similar capabilities.

\makevspace

Now consider a real valued function of $n$ variables $f:\mathbb{R}^n \to \mathbb{R}$.
The graph $G$ of $f$ is defined similarly to that of single variable functions,
\[
	G = \{ (\mathbf{x}, y) \in \mathbb{R}^n \times \mathbb{R} : y = f(\mathbf{x}) \}.	
\]
However, this graph is a subset of $\mathbb{R}^{n+1}$.
It is not obvious how one should visualize this graph.
A 3 dimensional world offers efficient visualization of two-variable functions at best. 
There are no well known calculators capable of visualizing a real function of $n$ variables.

\makevspace

The problem is that functions of $n$ variables are common in nature. 
It is hard to find a system where values are only dependent on one variable. 
We seek an application capable of visualzing the graph of functions of $n$ variables.
Such an application will offer details on functions where previously calculations had to 
be carried out by hand.

\newpage
\section{Solution}

\newpage
\section{Milestones \& Tasking}

\newpage
\section{MOSCOW}
\begin{tabularx}{\headerwidth}{|c|X|} 
	\hline
		  \multirow{5}{*}{ MUST: } 
		& { Method for displaying graphs of single variable real valued functions on screen } \\
	\cline{2-2}
		& { Method for storing real valued functions of $n$ variables } \\
	\cline{2-2}
		& { Method for displaying projections $n$ variable functions } \\
	\cline{2-2}
		& { Method for user to input $n$ variable functions } \\
	\cline{2-2}
		& { Run at stable 60fps with no warnings/errors } \\
	\hline
		\multirow{2}{*}{ SHOULD: }
		& { Parser to allow function input as text } \\
	\cline{2-2}
		& { Run on both Windows and Linux } \\
	\hline
		\multirow{1}{*}{ COULD: }
		& { Parser to allow function input as \LaTeX } \\
	\hline
		\multirow{1}{*}{ WOULD: }
		& {  } \\
	\hline
\end{tabularx}

} \end{document}